\documentclass[10pt,twocolumn,letterpaper]{article}

\usepackage{icb}
\usepackage{times}
\usepackage{epsfig}
\usepackage{graphicx}
\usepackage{amsmath}
\usepackage{amssymb}
\usepackage{eso-pic}

% Include other packages here, before hyperref.

% If you comment hyperref and then uncomment it, you should delete
% egpaper.aux before re-running latex.  (Or just hit 'q' on the first latex
% run, let it finish, and you should be clear).
%\usepackage[pagebackref=true,breaklinks=true,letterpaper=true,colorlinks,bookmarks=false]{hyperref}

\icbfinalcopy % *** Uncomment this line for the final submission

\def\icbPaperID{****} % *** Enter the IJCB Paper ID here
\def\httilde{\mbox{\tt\raisebox{-.5ex}{\symbol{126}}}}

% Pages are numbered in submission mode, and unnumbered in camera-ready
\ificbfinal\pagestyle{empty}\fi
\begin{document}

%%%%%%%%% TITLE
\title{Investigation of muscular architecture in ultrasound images}

\author{Thomas Bergmueller, Martin Schnoell\\
Medical Imaging LAB\\
Master program: Applied Image and Signal Processing\\
Fachhochschule Salzburg\\
%Institution1 address\\
{\tt\small tbergmueller.aise-m2013@fh-salzburg.ac.at, mschnoell.aise-m2013@fh-salzburg.ac.at}
% For a paper whose authors are all at the same institution,
% omit the following lines up until the closing ``}''.
% Additional authors and addresses can be added with ``\and'',
% just like the second author.
% To save space, use either the email address or home page, not both
%\and
%Second Author\\
%Institution2\\
%First line of institution2 address\\
%{\tt\small secondauthor@i2.org}
}

\maketitle
\thispagestyle{empty}

%%%%%%%%% ABSTRACT
%\begin{abstract}
%   The ABSTRACT is to be in fully-justified italicized text, at the top
%   of the left-hand column, below the author and affiliation
%   information. Use the word ``Abstract'' as the title, in 12-point
%   Times, boldface type, centered relative to the column, initially
%   capitalized. The abstract is to be in 10-point, single-spaced type.
%   Leave two blank lines after the Abstract, then begin the main text.
%   Look at previous ICB abstracts to get a feel for style and length.
%\end{abstract}

%%%%%%%%% BODY TEXT

\section{Introduction}
The main goal of this project was to write an algorithm which automatically calculates the angle between the aponeurosis and the muscle fibers of the Vastus Lateralis which is the muscle on the thigh right over the knee. This angle correlates with the constitution of the muscle.
For this project we used ultrasound images.  
Figure \ref{fig:VastusLateralis} shows the location of the Vastus Lateralis and an example ultrasound image in which both aponeuroses (lower and upper one) as well as the line is drawn which represents the muscle fibers (dashed line). Between this muscle fibers and the aponeurosis the angle should be calculated. In the most cases, the two aponeuroses are parallel, so there is no difference between taking the upper or lower one for the calculations. However, sometimes the two aponeuroses are not parallel and then, usually, the lower one is used. 


\begin{figure}
	\begin{center}		
		\includegraphics[width=1\linewidth]{img/VastusLateralis}
	\end{center}
	\caption{Vastus Lateralis (left) and an example ultrasound image in which both aponeuroses (lower and upper one) as well as the line is drawn which represents the muscle fibers (dashed line). \cite{NCronin13a}}
	\label{fig:VastusLateralis}
	
\end{figure}

Usually, this angle is calculated manually, by fitting lines to an ultrasound image on the computer and then calculate the angle by hand.

The available groundtruth consists of 22 ultrasound images of the Vastus Lateralis from 12 different patients.
Figure \ref{fig:im1_orig} shows one original image of the dataset.

\begin{figure}
	\begin{center}		
		\includegraphics[width=1\linewidth]{img/im1_original}
	\end{center}
	\caption{One original image of the dataset.}
	\label{fig:im1_orig}
	
\end{figure}

For every image two or three different measurements of the angle exist, which partly vary significantly. For example, is the biggest difference of two measurements for the same image (!) 5.7 degree (image with ID 1). In the most cases, however, the single measurements are in the range of +/- 1.5 to the mean.
For our implementation we calculated the mean angle of all measurements from one image and took this mean angle as a reference.

\section{Proposed algorithms}
We worked on two different algorithms in order to determine the angle. The first one is the Hough transform, which is a well-known algorithm for detecting lines in an image. The second one is based on template matching.

\subsection{Hough transform}
The Hough transform is a well-known and established algorithm for detecting lines and other shapes in images. For the implementation, we used MATLAB R2013a and the inbuilt MATLAB function hough.
This are the main steps of our implementation:

\begin{enumerate}
     \item Gamma correction for enhancing the white pixels
     \item Binarization of the image
     \item Hough transform applied directly on the binary image in order to detect the aponeurosis (the most prominent line)
     \item Canny edge detection on the binary image
     \item Hough transform on the edge image
     \item Finally: Calculate the mean angle of all muscle fiber candidates and then determine the difference between the angle of the aponeurosis and the mean. This results in the final angle
\end{enumerate}

For detecting the aponeurosis (step 3) we also restricted the hough function to only search for lines which are nearly horizontal ( +/- 10 degree roughly). This increases the probability for finding the correct aponeurosis. In step 6, finding the muscle fiber candidates means that the 25 most prominent lines are used. Here, the angle is also restricted to the range from ~6 to ~30 degree since almost all muscle fibers in the groundtruth are in this range and lines with other angles are then clearly outliers. Furthermore, depending on the position of the aponeurosis (at the top or bottom), the fiber candidates have to clearly also lay below (detected aponeurosis is at the top) or above (apo. at bottom) the aponeurosis.

Figure \ref{fig:im1_hough_apo} shows the detected aponeurosis and \ref{fig:im1_hough_fibers} shows the muscle fiber candidates on the image with ID 1 of the database.

\begin{figure}
	\begin{center}		
		\includegraphics[width=1\linewidth]{img/im1_hough_apo}
	\end{center}
	\caption{This image shows the detected aponeurosis (red line) of the hough transform approach.}
	\label{fig:im1_hough_apo}
	
\end{figure}

\begin{figure}
	\begin{center}		
		\includegraphics[width=1\linewidth]{img/im1_hough_fibers}
	\end{center}
	\caption{This image shows the 25 detected muscle fiber candidates from which the mean angle is taken.}
	\label{fig:im1_hough_fibers}
	
\end{figure}

\subsection{Template Matching}
As a second approach we introduce a method based on template matching. In template matching anything that serves as a model can be used to compute the similarity to a second instance of this model \cite{Brunelli09a}. The basic idea behind our template approach is that we can obtain the fascicle angle by moving around probe regions of the original image in a local neighbourhood of the probe location. Because fascicles can be seen as straight lines, the similarity reaches a maximum for movements along that line. Hence, by detecting which specific movements result in maximum similarity between probe region and a sample-region of equal size within it's local neighbourhood, we are going to detect the fascicle angle.


We take a rectangular sample region and compute similarity scores within a probe region. This process is illustrated in figure \ref{fig:templateApproach}. 

\begin{figure}
	\begin{center}		
		\includegraphics[width=1\linewidth]{img/templateApproach}
	\end{center}
	\caption{In the template approach a rectangular sample region (red) is extracted from the image. Then the sample is moved around within the probe region (green) and for each possible position the similarity score is computed.}
	\label{fig:templateApproach}
	
\end{figure}

As a similarity score we employ the pixel-wise absolute difference between each pixel in the probe region and the corresponding pixels of a sub-region (same size as the probe region) within the sample region.

In doing so, for each sample region a similarity score map is computed. Such sample similarity score maps are shown in figure \ref{fig:sampleAndSimScores}.  


\begin{figure}
	\begin{center}		
		\includegraphics[width=0.09\linewidth]{img/sample}
		\hspace{0.05\linewidth}
		\includegraphics[width=0.2\linewidth]{img/diff1}
		\includegraphics[width=0.2\linewidth]{img/diff2}
		\includegraphics[width=0.2\linewidth]{img/diff3}
		\includegraphics[width=0.2\linewidth]{img/diff4}
	\end{center}
	\caption{TODO}
	\label{fig:sampleAndSimScores}
	
\end{figure}


\section{Results}
The average error and distance to the groundtruth values for all 22 images of both approaches is as following: 

\begin{itemize}
     \item \emph{Hough transform}: \textbf{2.48 degree}
     \item \emph{Template matching}: \textbf{3.06 degree}
\end{itemize}

What has to be clearly considered, is the fact that the angles in the groundtruth are measured quite differently. So, if the groundtruth would have been constructed uniformly (with uniform ways of measuring the angle), we suppose that the results of our approaches would result in an even lower error.


\subsection{Discussion of Hough transform results}

With an average error of 2.48 degree the Hough transform performs overall really well and outperforms the template matching results. 
However, improvements could be clearly done in the aponeurosis detection since in the most cases the detected aponeurosis is just a purely horizontal line which a user would in some cases clearly not agree with. An example of such a situation can be seen in Figure \ref{fig:im7_hough_apo}. Here, the left part of the aponeurosis is clearly not horizontal, although this is the case on the right part. However, the user would expect that the detected line is closer to the left part of the aponeurosis.
The muscle fibers are giving in general really good results, so the strategy of taking 25 fiber candidates and average them in order to find the angle clearly pays off.

\begin{figure}
	\begin{center}		
		\includegraphics[width=1\linewidth]{img/im7_hough_apo}
	\end{center}
	\caption{Detected aponeurosis of image 7.}
	\label{fig:im7_hough_apo}
	
\end{figure}

\subsection{Discussion of template matching results}

...

\subsection{Comparison of both approaches}

Table \ref{tab:results} shows the groundtruth angles as well as the angles from both approaches for every single image.
Figure \ref{fig:errorPlot} shows an error plot of both strategies. As it can be seen in the plot, both algorithms do not always agree on a low or high distance to the groundtruth. So for some images like images 5, 19 and 20 the Hough transform clearly beats the template matching approach where e.g. for images 7 and 14 the template matching result is clearly the better one.
Image 15 is clearly the most challenging one and the corresponding groundtruth image (the image where the lines are drawn by hand from an expert user) is shown in Figure \ref{fig:im15_gt}. 
The hand-drawn lines are looking correct, although it can be seen that the original muscle fibers have some curvature, so the hand-drawn three fiber lines can clearly not cover such a curvature and represent more the left part of the fibers.
Figure \ref{fig:im15_hough_apo} and \ref{fig:im15_hough_fibers} are showing the Hough transform results and figure \ref{fig:im15_templ} the template matching results.
Both approaches seem to detect the correct lines and angles when looking at the corresponding images. So at least, both strategies cannot be considered as completely faulty in this case and it might be the case that just the groundtruth angle is not correct or is too far away from the really "true" angle.

\begin{figure}
	\begin{center}		
		\includegraphics[width=1\linewidth]{img/im15_gt}
	\end{center}
	\caption{Image 15 and the corresponding groundtruth lines.}
	\label{fig:im15_gt}
	
\end{figure}

\begin{figure}
	\begin{center}		
		\includegraphics[width=1\linewidth]{img/im15_hough_apo}
	\end{center}
	\caption{Image 15 and the detected aponeurosis from the Hough transform approach.}
	\label{fig:im15_hough_apo}
	
\end{figure}

\begin{figure}
	\begin{center}		
		\includegraphics[width=1\linewidth]{img/im15_hough_fibers}
	\end{center}
	\caption{Image 15 and the detected fiber candidates from the Hough transform approach.}
	\label{fig:im15_hough_fibers}
	
\end{figure}

\begin{figure}
	\begin{center}		
		\includegraphics[width=1\linewidth]{img/im15_templ}
	\end{center}
	\caption{Image 15 and the template matching result.}
	\label{fig:im15_templ}
	
\end{figure}

\begin{table}
\begin{center}
\begin{tabular}{| l | l | l | l |}
  \hline
  \emph{Image ID} & \emph{GT} & \emph{HT} & \emph{TM} \\ \hline
  1 & 13.73 & 12.82 & 18.43 \\ \hline
  2 & 16.07 & 13.39 & 17.24 \\ \hline
  3 & 11.70 & 13.34 & 10.56 \\ \hline
  4 & 12.00 & 13.53 & 11.50 \\ \hline
  5 & 13.55 & 10.64 & 7.10 \\ \hline
  6 & 11.60 & 14 & -666 \\ \hline
  7 & 17.20 & 11.34 & 16.83 \\ \hline
  8 & 10.63 & 12.27 & 12.32 \\ \hline
  9 & 11.17 & 11.16 & 8.49 \\ \hline
  10 & 16.47 & 10.86 & 11.19 \\ \hline
  11 & 12.10 & 11.93 & 15.12 \\ \hline
  12 & 13.35 & 13.56 & 11.90 \\ \hline
  13 & 13.57 & 15.09 & 11.65 \\ \hline
  14 & 19.50 & 14.67 & 20.56 \\ \hline
  15 & 24.90 & 14.57 & 17.35 \\ \hline
  16 & 13.23 & 12.87 & 9.59 \\ \hline
  17 & 14.20 & 12.57 & 12.90 \\ \hline
  18 & 13.75 & 12.66 & 9.03 \\ \hline
  19 & 17.30 & 13.83 & 9.42 \\ \hline
  20 & 17.97 & 14.66 & 11.63 \\ \hline
  21 & 10.37 & 11.73 & 9.51 \\ \hline
  22 & 12.30 & 11.19 & 13.49 \\ \hline
\end{tabular}
\caption{Comparison of the angles from the groundtruth (GT), the hough transform approach (GT) and from the template matching approach (TM).}
\label{tab:results}
\end{center}
\end{table}

\begin{figure}
	\begin{center}		
		\includegraphics[width=1\linewidth]{img/errorPlot2}
	\end{center}
	\caption{Error plot of both approaches.}
	\label{fig:errorPlot}
	
\end{figure}


\section{Conclusion}
Both strategies gave good results when looking at the distances to the groundtruth angles. Clearly, with a bigger and more reliable groundtruth both algorithms could be even more optimized.

{\small
\bibliographystyle{ieee}
\bibliography{egbib}
}

\end{document}
